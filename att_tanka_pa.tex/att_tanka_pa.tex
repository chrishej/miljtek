\documentclass[12pt,a4paper]{article}
\usepackage{array}
\usepackage[T1]{fontenc}
\usepackage[utf8]{inputenc}
\usepackage{caption}
\captionsetup[figure]{name=Figur}
\captionsetup[table]{name=Tabell}
\usepackage{graphicx}
\usepackage{amsmath}
\usepackage{amsfonts}
\usepackage{amssymb}
\usepackage{float}
\usepackage{alltt}
\usepackage{blindtext}
\author{Christoffer Hejdström}
\title{Läsning av artikel}

\begin{document}

\pagenumbering{gobble}
\maketitle
\newpage
\pagenumbering{arabic}

Lättläslighet:
\begin{itemize}
\item Är någon av artiklarna mer lättläslig, vad beror detta på?
\item Är vissa stycken lättare att ta till sig än andra, varför?
\item På vilka sätt har författarna försökt underlätta för läsaren att ta till sig information i artikeln?
\end{itemize}

Referenshantering:
\begin{itemize}
\item Vilken typ av referenshantering används och är referenshanteringen korrekt?
\item Finns det påståenden eller stycken som borde ha referenser men saknar?
\item Vilka delar är mest referensintensiva (delar med många referenser per stycke), finns det en
anledning?
\end{itemize}


Struktur:
\begin{itemize}
\item Återfinns alla klassiska strukturdelar i rapporten (Inledning, metod, resultat, diskussion och slutsats)?
\item Skulle man kunnat strukturera rapporten annorlunda, på vilket sätt?
\item Väcker titeln och rubrikerna läsarens intresse?
Tillvägagångssätt
\item Hur läste ni artikeln?
\item Var kan man söka/hitta artiklar?
\end{itemize}

Innehåll, Ellingsen och Aanondsen (2006)
\begin{itemize}
\item \item Vad för typ av frågeställning eller problem tror ni ligger bakom skrivandet av artikeln?
\item Varför valdes just kyckling som riktmärke för miljöprestanda i artikeln? Håller du med om
resonemanget?
\item Rapporten bedömer fisk och kyckling utifrån fler olika kategorier (landanvändning,
energianvändning, etc.), hur gör författarna för att beräkna den totala miljöpåverkan från de olika
födoämnena?
\end{itemize}

Innehåll, Huber (2000)
\begin{itemize}
\item \item Vad för typ av frågeställning eller problem tror ni ligger bakom skrivandet av artikeln?
\item Håller ni med Hubers resonemang om varför sufficiency och efficiency inte är önskvärda strategier för
att uppnå en hållbar utveckling?
\item Huber dedikerar en stor del av inledningen till att diskutera hur hållbarhet och hållbar utveckling
definierats och tolkas, håller du med Huber eller tycker du att han är för kritisk eller för godtrogen?
Ta gärna med exempel till seminariet!
\end{itemize}

\end{document}